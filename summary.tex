
%% bare_conf.tex
%% V1.4b
%% 2015/08/26
%% by Michael Shell
%% See:
%% http://www.michaelshell.org/
%% for current contact information.
%%
%% This is a skeleton file demonstrating the use of IEEEtran.cls
%% (requires IEEEtran.cls version 1.8b or later) with an IEEE
%% conference paper.
%%
%% Support sites:
%% http://www.michaelshell.org/tex/ieeetran/
%% http://www.ctan.org/pkg/ieeetran
%% and
%% http://www.ieee.org/

%%*************************************************************************
%% Legal Notice:
%% This code is offered as-is without any warranty either expressed or
%% implied; without even the implied warranty of MERCHANTABILITY or
%% FITNESS FOR A PARTICULAR PURPOSE! 
%% User assumes all risk.
%% In no event shall the IEEE or any contributor to this code be liable for
%% any damages or losses, including, but not limited to, incidental,
%% consequential, or any other damages, resulting from the use or misuse
%% of any information contained here.
%%
%% All comments are the opinions of their respective authors and are not
%% necessarily endorsed by the IEEE.
%%
%% This work is distributed under the LaTeX Project Public License (LPPL)
%% ( http://www.latex-project.org/ ) version 1.3, and may be freely used,
%% distributed and modified. A copy of the LPPL, version 1.3, is included
%% in the base LaTeX documentation of all distributions of LaTeX released
%% 2003/12/01 or later.
%% Retain all contribution notices and credits.
%% ** Modified files should be clearly indicated as such, including  **
%% ** renaming them and changing author support contact information. **
%%*************************************************************************


% *** Authors should verify (and, if needed, correct) their LaTeX system  ***
% *** with the testflow diagnostic prior to trusting their LaTeX platform ***
% *** with production work. The IEEE's font choices and paper sizes can   ***
% *** trigger bugs that do not appear when using other class files.       ***                          ***
% The testflow support page is at:
% http://www.michaelshell.org/tex/testflow/



\documentclass[conference]{IEEEtran}
% Some Computer Society conferences also require the compsoc mode option,
% but others use the standard conference format.
%
% If IEEEtran.cls has not been installed into the LaTeX system files,
% manually specify the path to it like:
% \documentclass[conference]{../sty/IEEEtran}





% Some very useful LaTeX packages include:
% (uncomment the ones you want to load)


% *** MISC UTILITY PACKAGES ***
%
%\usepackage{ifpdf}
% Heiko Oberdiek's ifpdf.sty is very useful if you need conditional
% compilation based on whether the output is pdf or dvi.
% usage:
% \ifpdf
%   % pdf code
% \else
%   % dvi code
% \fi
% The latest version of ifpdf.sty can be obtained from:
% http://www.ctan.org/pkg/ifpdf
% Also, note that IEEEtran.cls V1.7 and later provides a builtin
% \ifCLASSINFOpdf conditional that works the same way.
% When switching from latex to pdflatex and vice-versa, the compiler may
% have to be run twice to clear warning/error messages.






% *** CITATION PACKAGES ***
%
%\usepackage{cite}
% cite.sty was written by Donald Arseneau
% V1.6 and later of IEEEtran pre-defines the format of the cite.sty package
% \cite{} output to follow that of the IEEE. Loading the cite package will
% result in citation numbers being automatically sorted and properly
% "compressed/ranged". e.g., [1], [9], [2], [7], [5], [6] without using
% cite.sty will become [1], [2], [5]--[7], [9] using cite.sty. cite.sty's
% \cite will automatically add leading space, if needed. Use cite.sty's
% noadjust option (cite.sty V3.8 and later) if you want to turn this off
% such as if a citation ever needs to be enclosed in parenthesis.
% cite.sty is already installed on most LaTeX systems. Be sure and use
% version 5.0 (2009-03-20) and later if using hyperref.sty.
% The latest version can be obtained at:
% http://www.ctan.org/pkg/cite
% The documentation is contained in the cite.sty file itself.






% *** GRAPHICS RELATED PACKAGES ***
%
\ifCLASSINFOpdf
  % \usepackage[pdftex]{graphicx}
  % declare the path(s) where your graphic files are
  % \graphicspath{{../pdf/}{../jpeg/}}
  % and their extensions so you won't have to specify these with
  % every instance of \includegraphics
  % \DeclareGraphicsExtensions{.pdf,.jpeg,.png}
\else
  % or other class option (dvipsone, dvipdf, if not using dvips). graphicx
  % will default to the driver specified in the system graphics.cfg if no
  % driver is specified.
  % \usepackage[dvips]{graphicx}
  % declare the path(s) where your graphic files are
  % \graphicspath{{../eps/}}
  % and their extensions so you won't have to specify these with
  % every instance of \includegraphics
  % \DeclareGraphicsExtensions{.eps}
\fi
% graphicx was written by David Carlisle and Sebastian Rahtz. It is
% required if you want graphics, photos, etc. graphicx.sty is already
% installed on most LaTeX systems. The latest version and documentation
% can be obtained at: 
% http://www.ctan.org/pkg/graphicx
% Another good source of documentation is "Using Imported Graphics in
% LaTeX2e" by Keith Reckdahl which can be found at:
% http://www.ctan.org/pkg/epslatex
%
% latex, and pdflatex in dvi mode, support graphics in encapsulated
% postscript (.eps) format. pdflatex in pdf mode supports graphics
% in .pdf, .jpeg, .png and .mps (metapost) formats. Users should ensure
% that all non-photo figures use a vector format (.eps, .pdf, .mps) and
% not a bitmapped formats (.jpeg, .png). The IEEE frowns on bitmapped formats
% which can result in "jaggedy"/blurry rendering of lines and letters as
% well as large increases in file sizes.
%
% You can find documentation about the pdfTeX application at:
% http://www.tug.org/applications/pdftex





% *** MATH PACKAGES ***
%
%\usepackage{amsmath}
% A popular package from the American Mathematical Society that provides
% many useful and powerful commands for dealing with mathematics.
%
% Note that the amsmath package sets \interdisplaylinepenalty to 10000
% thus preventing page breaks from occurring within multiline equations. Use:
%\interdisplaylinepenalty=2500
% after loading amsmath to restore such page breaks as IEEEtran.cls normally
% does. amsmath.sty is already installed on most LaTeX systems. The latest
% version and documentation can be obtained at:
% http://www.ctan.org/pkg/amsmath





% *** SPECIALIZED LIST PACKAGES ***
%
%\usepackage{algorithmic}
% algorithmic.sty was written by Peter Williams and Rogerio Brito.
% This package provides an algorithmic environment fo describing algorithms.
% You can use the algorithmic environment in-text or within a figure
% environment to provide for a floating algorithm. Do NOT use the algorithm
% floating environment provided by algorithm.sty (by the same authors) or
% algorithm2e.sty (by Christophe Fiorio) as the IEEE does not use dedicated
% algorithm float types and packages that provide these will not provide
% correct IEEE style captions. The latest version and documentation of
% algorithmic.sty can be obtained at:
% http://www.ctan.org/pkg/algorithms
% Also of interest may be the (relatively newer and more customizable)
% algorithmicx.sty package by Szasz Janos:
% http://www.ctan.org/pkg/algorithmicx




% *** ALIGNMENT PACKAGES ***
%
%\usepackage{array}
% Frank Mittelbach's and David Carlisle's array.sty patches and improves
% the standard LaTeX2e array and tabular environments to provide better
% appearance and additional user controls. As the default LaTeX2e table
% generation code is lacking to the point of almost being broken with
% respect to the quality of the end results, all users are strongly
% advised to use an enhanced (at the very least that provided by array.sty)
% set of table tools. array.sty is already installed on most systems. The
% latest version and documentation can be obtained at:
% http://www.ctan.org/pkg/array


% IEEEtran contains the IEEEeqnarray family of commands that can be used to
% generate multiline equations as well as matrices, tables, etc., of high
% quality.




% *** SUBFIGURE PACKAGES ***
%\ifCLASSOPTIONcompsoc
%  \usepackage[caption=false,font=normalsize,labelfont=sf,textfont=sf]{subfig}
%\else
%  \usepackage[caption=false,font=footnotesize]{subfig}
%\fi
% subfig.sty, written by Steven Douglas Cochran, is the modern replacement
% for subfigure.sty, the latter of which is no longer maintained and is
% incompatible with some LaTeX packages including fixltx2e. However,
% subfig.sty requires and automatically loads Axel Sommerfeldt's caption.sty
% which will override IEEEtran.cls' handling of captions and this will result
% in non-IEEE style figure/table captions. To prevent this problem, be sure
% and invoke subfig.sty's "caption=false" package option (available since
% subfig.sty version 1.3, 2005/06/28) as this is will preserve IEEEtran.cls
% handling of captions.
% Note that the Computer Society format requires a larger sans serif font
% than the serif footnote size font used in traditional IEEE formatting
% and thus the need to invoke different subfig.sty package options depending
% on whether compsoc mode has been enabled.
%
% The latest version and documentation of subfig.sty can be obtained at:
% http://www.ctan.org/pkg/subfig




% *** FLOAT PACKAGES ***
%
%\usepackage{fixltx2e}
% fixltx2e, the successor to the earlier fix2col.sty, was written by
% Frank Mittelbach and David Carlisle. This package corrects a few problems
% in the LaTeX2e kernel, the most notable of which is that in current
% LaTeX2e releases, the ordering of single and double column floats is not
% guaranteed to be preserved. Thus, an unpatched LaTeX2e can allow a
% single column figure to be placed prior to an earlier double column
% figure.
% Be aware that LaTeX2e kernels dated 2015 and later have fixltx2e.sty's
% corrections already built into the system in which case a warning will
% be issued if an attempt is made to load fixltx2e.sty as it is no longer
% needed.
% The latest version and documentation can be found at:
% http://www.ctan.org/pkg/fixltx2e


%\usepackage{stfloats}
% stfloats.sty was written by Sigitas Tolusis. This package gives LaTeX2e
% the ability to do double column floats at the bottom of the page as well
% as the top. (e.g., "\begin{figure*}[!b]" is not normally possible in
% LaTeX2e). It also provides a command:
%\fnbelowfloat
% to enable the placement of footnotes below bottom floats (the standard
% LaTeX2e kernel puts them above bottom floats). This is an invasive package
% which rewrites many portions of the LaTeX2e float routines. It may not work
% with other packages that modify the LaTeX2e float routines. The latest
% version and documentation can be obtained at:
% http://www.ctan.org/pkg/stfloats
% Do not use the stfloats baselinefloat ability as the IEEE does not allow
% \baselineskip to stretch. Authors submitting work to the IEEE should note
% that the IEEE rarely uses double column equations and that authors should try
% to avoid such use. Do not be tempted to use the cuted.sty or midfloat.sty
% packages (also by Sigitas Tolusis) as the IEEE does not format its papers in
% such ways.
% Do not attempt to use stfloats with fixltx2e as they are incompatible.
% Instead, use Morten Hogholm'a dblfloatfix which combines the features
% of both fixltx2e and stfloats:
%
% \usepackage{dblfloatfix}
% The latest version can be found at:
% http://www.ctan.org/pkg/dblfloatfix




% *** PDF, URL AND HYPERLINK PACKAGES ***
%
%\usepackage{url}
% url.sty was written by Donald Arseneau. It provides better support for
% handling and breaking URLs. url.sty is already installed on most LaTeX
% systems. The latest version and documentation can be obtained at:
% http://www.ctan.org/pkg/url
% Basically, \url{my_url_here}.




% *** Do not adjust lengths that control margins, column widths, etc. ***
% *** Do not use packages that alter fonts (such as pslatex).         ***
% There should be no need to do such things with IEEEtran.cls V1.6 and later.
% (Unless specifically asked to do so by the journal or conference you plan
% to submit to, of course. )


% correct bad hyphenation here
\hyphenation{op-tical net-works semi-conduc-tor}

\usepackage{listings}


\begin{document}
%
% paper title
% Titles are generally capitalized except for words such as a, an, and, as,
% at, but, by, for, in, nor, of, on, or, the, to and up, which are usually
% not capitalized unless they are the first or last word of the title.
% Linebreaks \\ can be used within to get better formatting as desired.
% Do not put math or special symbols in the title.
\title{Why\\Functional Programming\\ Matters Even More}


% author names and affiliations
% use a multiple column layout for up to three different
% affiliations
\author{\IEEEauthorblockN{Olle Svensson}
\IEEEauthorblockA{Chalmers University of Technology\\
Gothenburg, Sweden\\
Email: ollesv@student.chalmers.se}}

% conference papers do not typically use \thanks and this command
% is locked out in conference mode. If really needed, such as for
% the acknowledgment of grants, issue a \IEEEoverridecommandlockouts
% after \documentclass

% for over three affiliations, or if they all won't fit within the width
% of the page, use this alternative format:
% 
%\author{\IEEEauthorblockN{Michael Shell\IEEEauthorrefmark{1},
%Homer Simpson\IEEEauthorrefmark{2},
%James Kirk\IEEEauthorrefmark{3}, 
%Montgomery Scott\IEEEauthorrefmark{3} and
%Eldon Tyrell\IEEEauthorrefmark{4}}
%\IEEEauthorblockA{\IEEEauthorrefmark{1}School of Electrical and Computer Engineering\\
%Georgia Institute of Technology,
%Atlanta, Georgia 30332--0250\\ Email: see http://www.michaelshell.org/contact.html}
%\IEEEauthorblockA{\IEEEauthorrefmark{2}Twentieth Century Fox, Springfield, USA\\
%Email: homer@thesimpsons.com}
%\IEEEauthorblockA{\IEEEauthorrefmark{3}Starfleet Academy, San Francisco, California 96678-2391\\
%Telephone: (800) 555--1212, Fax: (888) 555--1212}
%\IEEEauthorblockA{\IEEEauthorrefmark{4}Tyrell Inc., 123 Replicant Street, Los Angeles, California 90210--4321}}




% use for special paper notices
%\IEEEspecialpapernotice{(Invited Paper)}




% make the title area
\maketitle

% As a general rule, do not put math, special symbols or citations
% in the abstract
\begin{abstract}
The abstract goes here.
\end{abstract}

% no keywords




% For peer review papers, you can put extra information on the cover
% page as needed:
% \ifCLASSOPTIONpeerreview
% \begin{center} \bfseries EDICS Category: 3-BBND \end{center}
% \fi
%
% For peerreview papers, this IEEEtran command inserts a page break and
% creates the second title. It will be ignored for other modes.
\IEEEpeerreviewmaketitle



\section{Introduction}
% no \IEEEPARstart
% You must have at least 2 lines in the paragraph with the drop letter
% (should never be an issue)
The purpose of this paper is to argue why functional programming will continue to increase in popularity in the coming years and why its benefits will become even more significant. A functional language differs from conventional imperative languages like C++ or Java in that it consists solely of functions. A program is a function that receives arguments as its input and returns a result as the output. The program itself consists of even more functions that eventually reaches the lowest level of primitive types. Further, variables are immutable meaning that once given a value, it can never change. As a result, a functional language has no assignment statements. A functional language also does not have any side-effects. In other words, a function call cannot affect another part of the program during its execution and given a specific input, a function will always return the same value \cite{why}.

In 1989 Hughes published a paper with the title "Why Functional Programming Matters". In the paper, Hughes describes the significance and advantages of functional programming. He argues that a well-written program is a structured program. To ensure that this is achieved, the program should have a modular design. In turn, modular design is achieved by dividing the main problem into multiple smaller programs that solves subproblems. We call these smaller programs modules. The ability to create a well-written program consisting of multiple modules depends on the features a language offers to glue these modules together. Hughes proves his point by showing a few key features in functional languages that provide this kind of glue. One of these will be discussed in more detail in a following section, namely higher-order functions \cite{why}.

In 2015, Hughes revisited the topic together with Hu and Wang in a paper entitled "How Functional Programming Mattered". Hughes concludes together with his co-authors that he had been right 25 years earlier. Today, functional programming is no longer an academic phenomenon and functional languages can be found at many prominent companies in the industry \cite{how}. During this time there has also been notable change in the development of microprocessors. Since 1965 the clock speed of processors has increased according to Moore's law, but since around 2005 this is no longer the case. Instead, newer generations of processors are receiving multiple cores \cite{proc}. What is the reason for this change? How does it affect developers and the software they write? Most importantly, what does it mean for the future of functional programming? This is what we will try to answer in the following sections.


\section{Higher-Order Functions}
What differentiates higher-order functions from other functions is that they either take one or more functions as argument or that they return a function. They may also do both. We will look at a few examples from Hughes first paper to illustrate how these functions operate and to understand why Hughes considers them as a key feature of functional languages. We will use Haskell as our example language and since we will be working with lists we need to look at the definition of lists in Haskell:
\begin{lstlisting}[]
data List a = Nil | Cons a (List a)	
\end{lstlisting}
What the snippet above tells us is that a list of any arbitrary type $a$ is either $Nil$, meaning empty, or an element of type $a$ together with another list of type $a$. From an imperative perspective, one could say that the definition resembles a linked list in the sense that an element has a value and reference to the next value. Without going into detail of its meaning we will refer to this second case as $Cons$. We can define a list of three integers as

\begin{lstlisting}[]
listA = Cons 1 (Cons 2 (Cons 3 Nil))
\end{lstlisting}
but since this notation becomes cumbersome rather quickly we will use the more common notation of the elements within brackets separated by commas,

\begin{lstlisting}[]
listB     = [1,2,3]
listEmpty = []

\end{lstlisting}
which is also supported. 

Now that we have introduced lists we can move forward to defining functions. We define a recursive function $sum$ that sums a list of integers. Since a list can be both $Nil$ and $Cons$ we need to have a definition for both cases.

\begin{lstlisting}[]
sum []     = 0
sum (n:ns) = n + sum ns
\end{lstlisting}
Above we see an additional notation for lists. In the definition for $Cons$ there are two variables separated by a colon. With this notation the $n$ refers to the element or value and $ns$ refers to the rest of the list. To clarify our definition, the sum of an empty list is zero and otherwise the addition of the current element together with the sum of the rest of the list.

We define another function similar to $sum$ which computes the product instead. We define it as

\begin{lstlisting}[]
product []     = 1
product (n:ns) = n * product ns
\end{lstlisting}
When comparing the two functions we realise that the only difference between them is the value for the base case with an empty list and the binary operator for the general case. Following Hughes philosophy these two functions should be modularised into one general function solving both cases. Indeed there is such a function and it is called $foldr$. This also our first example of a higher-order function. The function $foldr$ takes a function, a starting value and a list and reduces the list into a single value by applying the function to each element together with the intermediate result. We can now define our previous functions in the following way 

\begin{lstlisting}[]
sum 	= foldr (+) 0
product = foldr (*) 1
\end{lstlisting}
The $foldr$-function takes three arguments but in our new definitions it is only given two. Further, neither $sum$ or $product$ takes the missing list as argument. The reason why this is possible is since $foldr$ is only given two arguments, Haskell automatically recognises $sum$ and $product$ as functions that takes a list as argument. The list is also passed to $foldr$ without having to define it explicitly. 

Another example where $foldr$ can be applied are lists of booleans

\begin{lstlisting}[mathescape=true]
anytrue = foldr ($\vee$) False
alltrue = foldr ($\wedge$) True
\end{lstlisting}
or computing the length of a list together with a small helper-function
\begin{lstlisting}[mathescape=true]
length 		= foldr count 0
count a n 	= n + 1
\end{lstlisting}
The helper-function $count$ will be called for each element in the list. It receives the current element and count in the variables $a$ and $n$. Since the actual value of the element is irrelevant $count$ simply ignores it and increments the current count with one. Since the starting value is zero and $count$ is called for each element in the list, the final result will be the length of the list.

Another example of a higher-order function is $map$ which takes a function and a list, applies the function to each element in the list and returns a list of the results. We could write a function that sums all elements in matrices as
\begin{lstlisting}[mathescape=true]
summatrix = sum . map sum
\end{lstlisting}
The function uses $map$ to apply $sum$ to each row and computes the total sum with the leftmost $sum$ \cite{why}.

The examples in this section are trivial and one could refer to Hughes original papers for more advanced examples. However, the main point has been to show how general patterns can be implemented as higher-order functions and how these can be combined with simple functions to create more complex functions. By this, we achieve the modularity and the powerful glue that we agreed is an important feature of a good programming language.

\section{Influence}
Today, functional languages have a widespread use in industry and many imperative languages have adopted features usually considered properties of functional languages. In fact, it is not unlikely many developers are implementing solutions using functional features without them realising it or having any knowledge of functional programming.

Starting with functional influence on other languages, lambda expression are supported in the latest versions of both C\# and C++. Type inference is also a part of C\#'s design. The generic type system introduced in Java 5 was inspired by the type system found in the functional language ML. Support was extended in Java 8 including lambda expressions and higher-order functions. Functional features are often found in multi-paradigm languages as well. The founder of Ruby admits having Lisp, a family of functional languages, as inspiration when creating Ruby. Both lambda expressions and list comprehensions are supported in Python. Further, the Python standard library offers many functions imported from Haskell and Standard ML.

One could say that functional programming already has a widespread use in industry by its existence in the languages already mentioned since they constitutes some of the most popular and widely used languages in the software industry. One example is the functional Java-library RxJava which is heavily used in the Netflix API. But there are also occurrences of purely functional languages. Facebook's system for filtering spam is written in Haskell. According to themselves it is the largest Haskell deployment in existence. Erlang, initially developed by Ericsson, is the language behind enterprises like Klarna AB and WhatsApp. Klarna AB offers invoice services for web shops in seven European countries and WhatsApp, a messaging application, was bought by Facebook in 2014 for a staggering 22 billion dollars \cite{how}.

\section{Evolution of microprocessors}

Since 1965 up until recently, the performance of microprocessors has been progressing according to Moore's law. This law was coined by the co-founder of Intel, Gordon Moore, who predicted that the number of components per integrated circuit would double every year for at least a decade. While the performance has increased in magnitudes, the structural architecture of microprocessors has received very few changes. This has been purposely avoided since the economic value of backwards compatibility is so strong. As a result, the architecture has consistently kept the form of a single processor executing a sequential stream of instructions with a large block of memory containing all data, at least in abstraction to software developers.

To increase the performance of microprocessors beyond Moore's law, with the restrictions regarding architecture, development has been focused on two different areas. One has been to enable the execution of multiple instructions in parallel and the other to increase clock frequency. To accommodate the first, superscalar processors were developed that could dynamically examine the sequential stream of instructions and rearrange them so that part of them could be executed in parallel. For example, if an instruction $i_{1}$ was being executed and it shared a number of logical steps with the following instruction $i_{2}$, these logical steps could be executed in parallel. In other words, the execution of $i_{2}$ could start before the execution of $i_{1}$ had finished. Obviously, these actions had to be performed without breaking the logic of the original program.

The use of superscalar processors was facilitated further by introducing the concept of pipelining. Existing instructions were sliced into smaller logical steps to reach a higher level of parallelism. An instruction that needed a few cycles to execute in 1980 might use 20 or more cycles in todays modern processors. Furthermore, since smaller logical steps meant less logic had to change in each cycle, the clock frequency could also be increased almost proportionally to the increase in cycles.

Focus on superscalar processors, pipelining and clock frequency has been successful since it has been possible to develop these areas without affecting developers and their approach in writing software. Unfortunately, this success has come to an end since there are limits in how much a certain area can evolve. Creating superscalar processors with support of executing more than four instructions in parallel provides a performance increase that is insignificant. Further, pipelining instructions into more than 30 steps would mean slicing the current minimal steps, such as adding two integers, into even smaller steps. This would add a significant increase in complexity that would not be justified in comparison to the performance gained. Finally, the improvement of clock frequency has reached its limit in form of power usage. The development has moved from processors in 1980 that used one watt without the need of a heat sink to modern processors using 100 watt with one or more dedicated fans to increase the airflow over the processor. If the power usage were to increase any further, fans would no longer be sufficient. The heat sinks needed in this case e.g. water cooling would be economically justifiable in very few systems.

Kunle Olukotun and Lance Hammond states in a paper called "The Future of Microprocessors" that there are only three dimensions in which processor performance can be increased beyond Moore's law, namely clock frequency, superscalar processors with pipelining and multiprocessing. Since the two first have reached their limit, there is only multiprocessing left to explore. This means that for the first time in roughly 40 years, software developers will have to adjust their approach in writing software if they want to be able to utilise the full potential of coming generations of multiprocessors \cite{proc}.

\section{Concurrency}

This section intends to give a brief comparison of concurrent programming in functional versus imperative languages. It will do so by referring to a paper by Steve Vinoski called "Concurrency in Erlang". Vinoski claims that writing correct multithreaded applications is beyond many developers technical abilities. The primary reason is that many imperative languages like Java and C++ uses a global state that is shared between threads. Access to this shared state must be handled carefully. Multiple threads allowed to perform read and write operations simultaneously can lead to corrupt data which in turn can lead to program instability. Needless to say, the risks of utilising multiple threads together with a shared state are well known and there are a variety of mechanisms for handling this e.g. using locks and semaphores. However, applying these mechanisms in a correct way is a non-trivial procedure. Using semaphores or locks on too large blocks of code will lead to poor performance and the behaviour of a single-threaded application since a majority of the code can only be accessed by one thread at a time. Making the blocks to small on the other hand greatly increases the risk of deadlocks by threads acquiring locks in different order.

To avoid this, one option is to leave the responsibility to a framework that either offer a correct way to serialise access to a shared state or by avoiding a shared state altogether. Though avoiding state altogether might mean skewing the language in a way it was not intentionally meant to be used, and the suitability of such a solution could be questioned. Further, avoiding state also requires great discipline from the developer since neither the framework or the language itself can prevent the developers from accidentally introducing their own errors anyway. Vinoski suggests that a perhaps better solution would be to use a language specifically design to not use the concept of state, which leads us to functional language Erlang.

Erlang was initially developed at Ericsson to create a highly concurrent, fault tolerant and reliable language for their telecommunications switches. It also has an important functional feature in the form of immutable objects. Unlike Java or C++, the value of a variable in Erlang cannot change once it has been initialised. Instead of updating values through assignment statements, the equality operator in Erlang either initialises a variable or it becomes a boolean expression. If the variable on the left-hand side is unbound, it will receive the value on the right-hand side. If the left-hand variable is already initialised the boolean expression will return true if both values are equal and false otherwise.

What makes immutable objects a powerful feature is that there is no longer any need to worry about read and write operations occurring simultaneously. Since values are immutable they do not need concurrency protection. Further, since Erlang was design to be highly concurrent, its threads or processes are very lightweight. Vinoski conducted a simple test where Erlang was able to launch one million processes in 0.51 seconds. In comparison, C++ on the same machine was able to create 7,044 threads in 1.3 seconds before returning an error due to lack of resources. The test was extended to Java 5 as well which created 431,430 threads in 83.3 seconds before running out of resources \cite{con}.

Finally, it should be noted that the only functional language mentioned in this section is Erlang, but the benefits of e.g. immutable objects applies to functional languages in general.

%\subsection{Subsection Heading Here}
%Subsection text here.


%\subsubsection{Subsubsection Heading Here}
%Subsubsection text here.


% An example of a floating figure using the graphicx package.
% Note that \label must occur AFTER (or within) \caption.
% For figures, \caption should occur after the \includegraphics.
% Note that IEEEtran v1.7 and later has special internal code that
% is designed to preserve the operation of \label within \caption
% even when the captionsoff option is in effect. However, because
% of issues like this, it may be the safest practice to put all your
% \label just after \caption rather than within \caption{}.
%
% Reminder: the "draftcls" or "draftclsnofoot", not "draft", class
% option should be used if it is desired that the figures are to be
% displayed while in draft mode.
%
%\begin{figure}[!t]
%\centering
%\includegraphics[width=2.5in]{myfigure}
% where an .eps filename suffix will be assumed under latex, 
% and a .pdf suffix will be assumed for pdflatex; or what has been declared
% via \DeclareGraphicsExtensions.
%\caption{Simulation results for the network.}
%\label{fig_sim}
%\end{figure}

% Note that the IEEE typically puts floats only at the top, even when this
% results in a large percentage of a column being occupied by floats.


% An example of a double column floating figure using two subfigures.
% (The subfig.sty package must be loaded for this to work.)
% The subfigure \label commands are set within each subfloat command,
% and the \label for the overall figure must come after \caption.
% \hfil is used as a separator to get equal spacing.
% Watch out that the combined width of all the subfigures on a 
% line do not exceed the text width or a line break will occur.
%
%\begin{figure*}[!t]
%\centering
%\subfloat[Case I]{\includegraphics[width=2.5in]{box}%
%\label{fig_first_case}}
%\hfil
%\subfloat[Case II]{\includegraphics[width=2.5in]{box}%
%\label{fig_second_case}}
%\caption{Simulation results for the network.}
%\label{fig_sim}
%\end{figure*}
%
% Note that often IEEE papers with subfigures do not employ subfigure
% captions (using the optional argument to \subfloat[]), but instead will
% reference/describe all of them (a), (b), etc., within the main caption.
% Be aware that for subfig.sty to generate the (a), (b), etc., subfigure
% labels, the optional argument to \subfloat must be present. If a
% subcaption is not desired, just leave its contents blank,
% e.g., \subfloat[].


% An example of a floating table. Note that, for IEEE style tables, the
% \caption command should come BEFORE the table and, given that table
% captions serve much like titles, are usually capitalized except for words
% such as a, an, and, as, at, but, by, for, in, nor, of, on, or, the, to
% and up, which are usually not capitalized unless they are the first or
% last word of the caption. Table text will default to \footnotesize as
% the IEEE normally uses this smaller font for tables.
% The \label must come after \caption as always.
%
%\begin{table}[!t]
%% increase table row spacing, adjust to taste
%\renewcommand{\arraystretch}{1.3}
% if using array.sty, it might be a good idea to tweak the value of
% \extrarowheight as needed to properly center the text within the cells
%\caption{An Example of a Table}
%\label{table_example}
%\centering
%% Some packages, such as MDW tools, offer better commands for making tables
%% than the plain LaTeX2e tabular which is used here.
%\begin{tabular}{|c||c|}
%\hline
%One & Two\\
%\hline
%Three & Four\\
%\hline
%\end{tabular}
%\end{table}


% Note that the IEEE does not put floats in the very first column
% - or typically anywhere on the first page for that matter. Also,
% in-text middle ("here") positioning is typically not used, but it
% is allowed and encouraged for Computer Society conferences (but
% not Computer Society journals). Most IEEE journals/conferences use
% top floats exclusively. 
% Note that, LaTeX2e, unlike IEEE journals/conferences, places
% footnotes above bottom floats. This can be corrected via the
% \fnbelowfloat command of the stfloats package.




\section{Conclusion}

This paper has introduced the concept of higher-order functions in functional languages. Further, it has shown that this concept has been adopted by many of the most prominent imperative languages currently in use. Functional languages have gone from being used mainly for teaching purposes to being used by enterprises such as Facebook. At same time, the development of microprocessors has reached a point where it can only continue in one direction, namely multiprocessing. As a consequence, software developers will be forced to write software with concurrency support to be able to utilise the full potential of coming generations of processors. In turn, concurrency happens to be an area where imperative languages struggle and where functional languages excel.

What the sentences above tells us is that functional programming has evolved in an incredible way in 25 years and that it might come to play an even bigger role in software development in the next 25 years.




% conference papers do not normally have an appendix

% trigger a \newpage just before the given reference
% number - used to balance the columns on the last page
% adjust value as needed - may need to be readjusted if
% the document is modified later
%\IEEEtriggeratref{8}
% The "triggered" command can be changed if desired:
%\IEEEtriggercmd{\enlargethispage{-5in}}

% references section

% can use a bibliography generated by BibTeX as a .bbl file
% BibTeX documentation can be easily obtained at:
% http://mirror.ctan.org/biblio/bibtex/contrib/doc/
% The IEEEtran BibTeX style support page is at:
% http://www.michaelshell.org/tex/ieeetran/bibtex/
%\bibliographystyle{IEEEtran}
% argument is your BibTeX string definitions and bibliography database(s)
%\bibliography{IEEEabrv,../bib/paper}
%
% <OR> manually copy in the resultant .bbl file
% set second argument of \begin to the number of references
% (used to reserve space for the reference number labels box)



\begin{thebibliography}{1}

% [2] L. Stein, “Random patterns,” in Computers and You, J. S. Brake, Ed. New York: Wiley, 1994, pp. 55-70.

% [1] J. K. Author, “Title of chapter in the book,” in Title of His Published Book, xth ed. City of Publisher, Country if not
% USA: Abbrev. of Publisher, year, ch. x, sec. x, pp. xxx–xxx. 

% \bibitem{IEEEhowto:kopka}
% H.~Kopka and P.~W. Daly, \emph{A Guide to \LaTeX}, 3rd~ed.\hskip 1em plus
% 0.5em minus 0.4em\relax Harlow, England: Addison-Wesley, 1999.

\bibitem{why}
J. ~Hughes, "Why Functional Programming Matters," in Research Topics in Functional Programming, D. Turner, Ed. Boston: Addison-Wesley, 1990, pp. 17-42.

\bibitem{how}
Z. Hu et al., "How Functional Programming Mattered," in National Science Review, vol. 2, no. 3, pp. 349-367, July 2015.

\bibitem{con}
S. Vinoski, "Concurrency with Erlang," in IEEE Internet Computing, vol. 11, no. 5, pp. 90-93, September 2007.

\bibitem{proc}
K. Olukotun. and L. Hammond, "The Future of Microprocessors," in ACM Queue, vol. 3, no. 7, pp. 27-34, September 2005.

\end{thebibliography}






% that's all folks
\end{document}


